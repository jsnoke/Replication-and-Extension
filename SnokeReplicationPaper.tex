%%%%%%%%%%%%%%%%%%%%%%%%%%%%%%%%%%%%%%%%%%%%%%%%%%%%%%%%%%%%%%%%%%%%%%%%%%%%%%%%%%%%%%%%%%
\documentclass[12pt]{article} %a4paper

\usepackage{graphicx} % to include pictures
\usepackage{subfig} % create subfigures within figures
\usepackage{pdflscape} % e.g. to rotate one page of the document
\usepackage{booktabs} % make better looking tabels with different line types and
%stuff
\usepackage[left=2.5cm,right=3cm,top=3cm,bottom=2.5cm]{geometry}
\usepackage{fancyhdr} % for pages with custom headers and footers
\usepackage[utf8]{inputenc}
\usepackage{float}
\usepackage{datetime}
\usepackage{natbib}
\bibliographystyle{unsrtnat}
\usepackage{setspace}
\usepackage{amsmath}
\usepackage{hyperref}
\usepackage{verbatim}
%\usepackage{pa}

\setlength{\parindent}{0.0in}
\setlength{\parskip}{1ex plus 0.5ex minus 0.2ex}
\mmddyyyydate
%%%%%%%%%%%%%%%%%%%%%%%%%%%%%%%%%%%%%%%%%%%%%%%%%%%%%%%%%%%%%%%%%%%%%%%%%%%%%%%%%%%%%%%%%%


\usepackage[english]{babel}															% English language/hyphenation
\usepackage[protrusion=true,expansion=true]{microtype}	
\usepackage{amsmath,amsfonts,amsthm} % Math packages
%\usepackage[pdftex]{graphicx}	
%\usepackage{verbatim}
\usepackage{url}
%\usepackage{float}


%%% Custom sectioning
\usepackage{sectsty}
\allsectionsfont{\centering \normalfont\scshape}


%%% Custom headers/footers (fancyhdr package)
\usepackage{fancyhdr}
\pagestyle{fancyplain}
\fancyhead{}											% No page header
\fancyfoot[L]{}											% Empty 
\fancyfoot[C]{}											% Empty
\fancyfoot[R]{\thepage}									% Pagenumbering
\renewcommand{\headrulewidth}{0pt}			% Remove header underlines
\renewcommand{\footrulewidth}{0pt}				% Remove footer underlines
\setlength{\headheight}{13.6pt}


%%% Equation and float numbering
\numberwithin{equation}{section}		% Equationnumbering: section.eq#
\numberwithin{figure}{section}			% Figurenumbering: section.fig#
\numberwithin{table}{section}				% Tablenumbering: section.tab#


%%% Maketitle metadata
\newcommand{\horrule}[1]{\rule{\linewidth}{#1}} 	% Horizontal rule


\title{
		%\vspace{-1in} 	
		\usefont{OT1}{bch}{b}{n}
		\normalfont \normalsize \textsc{Pennsylvania State University} \\ [25pt]
            \normalfont \normalsize \textsc{PLSC 597E} \\ [25pt]
		%\horrule{0.5pt} \\[0.2cm]
		\Large Replication Excercise -  Self-Organizing Policy Networks: Risk, Partner, Selection, and Cooperation in Estuaries\\
		%\horrule{2pt} \\[0.5cm]
}

\author{
		\normalfont 								\normalsize
        Joshua Snoke \\[-2pt]		\normalsize
}
\date{}


%%% Begin document
\begin{document}
\maketitle

\section*{Update}
I have switched replication studies to "Self-Organizing Policy Networks: Risk, Partner, Selection, and Cooperation in Estuaries." by Berardo and Ramiro. This paper is of interest due to its use of two waves of longitudinal data with missingess in the second wave. The original paper did nothing to account for this missingness. 

I have obtained the complete data, and have replicated most of the summary statistics from Table A2 in the original paper. (The exception being out and in 2-stars - these are not clearly defined and I'm unsure what they mean.) I have yet to run the ERGM model or the MCMC method described. These could present potential difficulties given they do not define their models very specifically, but they do at least give a decent idea.

\section*{Citation}
Berardo, Ramiro, and John T. Scholz. "Self-Organizing Policy Networks: Risk, Partner, Selection, and Cooperation in Estuaries." American Journal of Political Science 54, no. 3 (2010):632-649.

\section*{Data and Code}
\url{https://github.com/jsnoke/Replication-and-Extension}

\section*{Network Visuals}
Below is a visualization for wave 1. (Working on getting isolates grouped correctly.)

\begin{figure}[!ht]
      \includegraphics[width=\linewidth]{wave1Plot.pdf}
      \caption{Wave 1 network}\label{fig:x1}
\end{figure}


\end{document}


